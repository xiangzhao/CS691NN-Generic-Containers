%% LyX 2.0.0rc3 created this file.  For more info, see http://www.lyx.org/.
%% Do not edit unless you really know what you are doing.
\documentclass[twocolumn,english,natbib]{sigplanconf}
\usepackage[T1]{fontenc}

\makeatletter

%%%%%%%%%%%%%%%%%%%%%%%%%%%%%% LyX specific LaTeX commands.
\newcommand{\noun}[1]{\textsc{#1}}

\makeatother

\usepackage{babel}
\begin{document}

\CopyrightYear{2011}


\title{Adaptivity in STL sequence data structures}


\authorinfo{Eli Gottlieb\and Xiang Zhao}{University of Massachusetts Amherst}{egottlie@student.umass.edu\\
xiang@cs.umass.edu}
\maketitle
\begin{abstract}
The C++ Standard Template Library provides a variety of sequence data
structures, each of which has different performance characteristics.
Performing the wrong operations upon a sequence \emph{en masse} can
result in major performance penalties. We have therefore created an
\emph{adaptive} sequence in C++ that changes its internal representation
to the optimal data structure for the operations being performed upon
it. We present the adaptive sequence's design and evaluate its performance
characteristics when tested against the original C++ STL.
\end{abstract}

\terms{Algorithms, Performance, Design}


\section{Introduction}

The C++ Standard Template Library is one of the most widely used software
libraries in the world. For every \emph{concept} in the library, multiple
implementations are provided as concrete classes. The sequence concept
is provided in the vector, (doubly-linked) list, and deque (double-ended
queue) forms. Each of these has different performance characteristics
both in terms of asymptotic time complexity and memory locality, and
serious losses in performance have been observed\cite{Liu2009} due
to using the wrong data structure.

In the ideal case, the programmer should be able to simply ask for
a sequence and not have to worry about the precise implementation.
We have attempted to take a step towards that condition by building
an \emph{adaptive} sequence. Our adaptive sequence logs the method-calls
it receives and uses that history to \emph{adapt} its internal data
structure, changing to a list, vector, or deque for the optimal asymptotic
time complexity.


\subsection{Related Work}

Liu et al\cite{Liu2009} found that choosing the right STL data structure
can result in as much as a 17\% speed improvement, with many smaller
improvements being found. Their code analyzer did not allow for generic
treatment of sequences without regard to their implementation but
required that the code only employ the methods shared between all
the STL sequence classes.

A group of researchers set out to use machine-learning techniques
to decide what data structure should be used where, producing software
they called Brainy\cite{Jung2011}. Brainy used empirical time and
space measurements to determine the real relative performance of different
set implementations, and thus trained a neural network to analyze
code and determine what data structures to use. Brainy trained its
neural-network using randomly-generated applications built by its
\emph{application generator}, software that would have proved extremely
useful in benchmarking our own adaptive sequences. Brainy achieved
27\% and 33\% speed-ups on two different test machines.


\section{Design and API support}

Our adaptive sequence supports most of the combined method set of
STL lists, vectors, and deques. It currently only lacks a small number
of methods and the production of random-access iterators rather than
bidirectional iterators. It can therefore be used as an imperfect
but workable drop-in replacement for many, if not most, uses of STL
sequences.

The adaptive sequence logs the methods called upon it, and upon each
operation calculates the amortized complexity of having performed
the entire logged set of operations with each possible internal data
structure. The internal STL vector, list, or deque itself is stored
as a tagged variant, on which tag dispatch is done to execute methods
upon the internal data structure. When a logged set of operations
could have been done more quickly with an internal data structure
besides the one currently in use (accounting for a full iteration
over the sequence to convert it), the internal data-structure is \emph{adapted}
to the optimal alternative; all data elements are moved into the new
internal data structure while preserving their locations and order.

The drawback to our API implementation is that, since we only support
bidirectional iterators rather than random-access iterators, a number
of the STL \emph{algorithms} (a misnomer, as these are templated library
functions implementing common algorithms rather than just theoretical
algorithms) do not work on the adaptive sequence at this time. Each
iterator operation can also only count for a single constant-time
operation alone. Such operations are not logged because this would
result in logging N constant-time \noun{ACCESS\_ELEMENT} operations
for a full iteration over a sequence of N elements rather than logging
an \noun{ITERATE\_OVER} operation. This would throw off the accuracy
of our amortized-complexity calculations.


\section{Performance evaluation}


\section{Conclusions}
\begin{acks}
We would like to thank ...\end{acks}
\begin{thebibliography}{References}
\bibitem{Liu2009}L. Liu and S. Rus. perflint: A Context Sensitive
Performance Advisor for C++ Programs. In \emph{Proceedings of the
2009 International Symposium on Code Generation and Optimization},
March 2009.

\bibitem{Jung2011}Changhee Jung, Silvius Rus, Brian P. Railing, Nathan
Clark, and Santosh Pande. Brainy: Effective Selection of Data Structures.
In \emph{Proceedings of the 2011 ACM conference on Programming Language
Design and Implementation}, preprint, retrieved May 2011.   \end{thebibliography}

\end{document}
