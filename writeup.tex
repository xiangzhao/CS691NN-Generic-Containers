%% LyX 2.0.0rc3 created this file.  For more info, see http://www.lyx.org/.
%% Do not edit unless you really know what you are doing.
\documentclass[twocolumn,english,natbib]{sigplanconf}
\usepackage[T1]{fontenc}
\usepackage{babel}
\begin{document}

\CopyrightYear{2011}


\title{Adaptivity in STL sequence data structures}


\authorinfo{Eli Gottlieb\and Xiang Zhao}{University of Massachusetts Amherst}{egottlie@student.umass.edu\\
xiang@cs.umass.edu}
\maketitle
\begin{abstract}
This paper is a demonstration of the Lyx template for the ACM alternate
latex style file. It works quite well and makes writing an ACM paper
so much easier.
\end{abstract}

\terms{Algorithms, Performance, Design}


\section{Introduction}


\subsection{Related Work}
\begin{enumerate}
\item test
\item test
\end{enumerate}

\section{Design and API support}


\section{Performance evaluation}


\section{Conclusions}
\begin{acks}
We would like to thank ...\end{acks}
\begin{thebibliography}{References}
\bibitem{ahmed02}Amal~J. Ahmed, Andrew~W. Appel, and Roberto Virga.\newblock
A stratified semantics of general references embeddable in higher-order
logic.\newblock I\emph{n Proceedings of the 17th Annual IEEE Symposium
on Logic in Computer Science (LICS 2002)}, July 2002.

\bibitem{appel01:fpcc}Andrew~W. Appel.\newblock Foundational proof-carrying
code.\newblock In \emph{Symposium on Logic in Computer Science (LICS
\textquoteright{}01)}, pages 247--258. IEEE, 2001.\end{thebibliography}

\end{document}
